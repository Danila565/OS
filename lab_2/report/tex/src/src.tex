\section{Общие сведения о программе}

Программа компилируется из файла main.c. Подключены заголовочные файлы: unistd.h, fcntl.h, stdlib.h. В программе используются следующие системные вызовы:

\begin{enumerate}
    \item \textbf{pipe} –– принимает массив из двух целых чисел, в случае успеха массив будет содержать два файловых дескриптора, которые будут использоваться для конвейера, первое число в массиве предназначено для чтения, второе для записи, а так же вернется 0. В случае неуспеха вернется -1.
    \item \textbf{fork} –– создает новый процесс, который является копией родительского процесса, за исключением разных process ID и parent process ID. В случае успеха fork() возвращает 0 для ребенка, число больше 0 для родителя – child ID, в случае ошибки возвращает -1.
    \item \textbf{open} –– создает или открывает файл, если он был создан. В качестве аргументов принимает путь до файла, режим доступа (запись, чтение и т.п.),  модификатор доступа ( при создании можно указать права для файла ). Возвращает в случае успеха файловый дескриптор – положительное число, иначе возвращает -1.
    \item \textbf{close} –– принимает файловый дескриптор в качестве аргумента, удаляет файловый дескриптор из таблицы дескрипторов, в случае успеха вернет 0, в случае неуспеха вернет -1.
    \item \textbf{read} –– предназначена для чтения какого-то числа байт из файла, принимает в качестве аргументов файловый дескриптор, буфер, в который будут записаны данные и число байт. В случае успеха вернет число прочитанных байт, иначе -1.
    \item \textbf{write} –– предназначена для записи какого-то числа байт в файл, принимает в качестве аргументов файловый дескриптор, буфер, из которого будут считаны данные для записи и число байт. В случае успеха вернет число записанных байт, иначе -1.
\end{enumerate}

\pagebreak

\section{Общий метод и алгоритм решения}

Для реализации поставленной задачи необходимо:

\begin{enumerate}
    \item Изучить принципы работы pipe и fork.
    \item Написать функцию считывания имён выходных файлов
    \item Создать каналы связи для каждого из дочерних процессов
    \item Создать функцию обработки ввода
    \item Создать функцию фильтрации в родительском процессе
    \item Создать функцию фильтрации в дочерних процессах
    \item Написать обработку ошибок
    \item Написать тесты
\end{enumerate}

\pagebreak

\section{Исходный код}

\textbf{main.c}

\begin{lstlisting}[language=C]

#include <fcntl.h>
#include <stdlib.h>
#include <stdio.h>
#include <unistd.h>

void child_work(int from, int to) {
	char buf[1];
	while (read(from, buf, 1) > 0) {
		char c = buf[0];
		if (c != 'a' && c != 'e' && c != 'i' && c != 'o' && c != 'u' && c != 'y' &&
		c != 'A' && c != 'E' && c != 'I' && c != 'O' && c != 'U' && c != 'Y') {
			write(to, buf, 1);
		}
	}
	close(to);
	close(from);
}

void parrent_work(int child1, int child2) {
	char buf[1];
	int is_even = 0;
	while (read(STDIN_FILENO, buf, 1) > 0) {
		if (!is_even) {
			write(child1, buf, 1);
		} else {
			write(child2, buf, 1);
		}
		if (buf[0] == '\n') {
			is_even = !is_even;
		}
	}
	
	close(child1);
	close(child2);
}

int open_file() {
	const size_t NAME_SIZE = 64;
	char f_name[NAME_SIZE];
	char buf[1];
	int idx = 0;
	while (idx < NAME_SIZE && read(STDIN_FILENO, buf, 1) > 0) {
		if (buf[0] == '\n') {
			break;
		}
		f_name[idx++] = buf[0];
	}
	f_name[idx] = '\0';
	return open(f_name, O_WRONLY | O_TRUNC);
}

int main(int argc, char* argv[]) {
	int f1 = open_file();
	if (f1 == -1) {
		perror("File not found");
		exit(1);
	}
	int f2 = open_file();
	if (f2 == -1) {
		perror("File not found");
		exit(2);
	}
	
	int pipefd1[2];
	if (pipe(pipefd1) == -1) {
		perror("Cannot create pipe");
		exit(3);
	}
	
	int child1 = fork();
	if (child1 == -1) {
		perror("Can not create process");
		exit(4);
	}
	if (child1 == 0) {
		close(pipefd1[1]);
		child_work(pipefd1[0], f1);
		return 0;
	}
	close(pipefd1[0]);
	
	int pipefd2[2];
	
	if (pipe(pipefd2) == -1) {
		perror("Cannot create pipe");
		exit(5);
	}
	
	int child2 = fork();
	if (child2 == -1) {
		perror("Can not create process");
		exit(6);
	}
	if (child2 == 0) {
		close(pipefd1[1]);
		close(pipefd2[1]);
		child_work(pipefd2[0], f2);
		return 0;
	}
	close(pipefd2[0]);
	
	parrent_work(pipefd1[1], pipefd2[1]);
	
	return 0;
}

\end{lstlisting}

\pagebreak