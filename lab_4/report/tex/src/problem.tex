\section{Постановка задачи}

{\bfseries Цель работы:} 

Приобретение практических навыков в:

\begin{itemize}
    \item Освоение принципов работы с файловыми системами
    \item Обеспечение обмена данных между процессами посредством технологии «File mapping»
\end{itemize}

{\bfseries Задание (вариант 18):} 

Составить и отладить программу на языке Си, осуществляющую работу с процессами и взаимодействие между ними в одной из двух операционных систем. В результате работы программа (основной процесс) должен создать для решение задачи
один или несколько дочерних процессов. Взаимодействие между процессами осуществляется через системные сигналы/события и/или через отображаемые файлы (memory-mapped files). Необходимо обрабатывать системные ошибки, которые могут
возникнуть в результате работы.

Родительский процесс создает два дочерних процесса. Первой строкой пользователь в консоль родительского процесса вводит имя файла, которое будет использовано для открытия File с таким именем на запись для child1. Аналогично для второй строки и процесса child2. Родительский и дочерний процесс должны быть представлены разными программами.

Правило фильтрации: нечетные строки отправляются child1, четные child2. Дочерние процессы удаляют все гласные из строк.

\pagebreak
